%#! 使い方:
% * 一度にタイプセットする場合:
%    $ platex -shell-escape example-ja.tex     ※これは [extract] オプションなし
%    $ dvipdfmx example-ja.dvi
% * より安全にタイプセットしたい場合:
%    $ platex example-ja.tex                   ※このときは [extract] オプション付き \usepackage[extract]{chemobabel}
%    $ pdflatex -shell-escape ChemFigFile.tex  ※自動生成する ChemFigFile.tex は pdflatex 用ソース
%    $ platex example-ja.tex                   ※今度は [extract] オプションを付けない
%    $ dvipdfmx example-ja.dvi

\documentclass[dvipdfmx]{jsarticle}
\usepackage[librsvg]{chemobabel}

\title{\textsf{chemobabel.sty} の使用例}
\author{アセトアミノフェン}
\begin{document}
\maketitle
以下の構造式は、\textsf{chemobabel} パッケージを用いて Open Babel と Inkscape を介した自動変換により生成しました。

\begin{figure}[ht]
  \centering
  \smilesobabel{CCO}{}
  \caption{エタノール}
\end{figure}

\begin{figure}[ht]
  \centering
  \smilesobabel[scale=0.8]{C1[C@H](C)C[C@@H](O)C1}{}
  \caption{(1\textit{S},3\textit{S})-3-メチルシクロペンタノール}
\end{figure}

\begin{figure}[ht]
  \centering
  \smilesobabel[scale=0.7]{Cl/C=C/Br}{}
  \smilesobabel[scale=0.7]{Cl/C=C\Br}{}
  \caption{(\textit{E})-1-ブロモ-2-クロロエテンと(\textit{Z})-1-ブロモ-2-クロロエテン}
\end{figure}

\begin{figure}[ht]
  \centering
  \chemobabel[scale=0.6]{draw/ATP.cdx}{-xd}
  \caption{ATP (アデノシン三リン酸)}
\end{figure}

\begin{figure}[ht]
  \centering
  \chemobabel[scale=0.4]{draw/Brevetoxin A.mol}{}
  \caption{ブレベトキシンA}
\end{figure}

\end{document}