%#! Usage:
% * To typeset all at once:
%    $ pdflatex -shell-escape example-en.tex   (without ``extract'' option)
% * To typeset more safely:
%    $ pdflatex example-en.tex                 (with ``extract'' option: \usepackage[extract]{chemobabel})
%    $ pdflatex -shell-escape ChemFigFile.tex  (auto-generated pdfLaTeX source)
%    $ pdflatex example-en.tex                 (without ``extract'' option)

\documentclass{article}
\usepackage[librsvg]{chemobabel}

\title{An Example for \textsf{chemobabel} Package}
\author{Acetaminophen}
\begin{document}
\maketitle

The formulas below are generated using \textsf{chemobabel} package with the help of Open Babel and Inkscape.

\begin{figure}[ht]
  \centering
  \smilesobabel{CCO}{}
  \caption{Ethanol}
\end{figure}

\begin{figure}[ht]
  \centering
  \smilesobabel[scale=0.8]{C1[C@H](C)C[C@@H](O)C1}{}
  \caption{(1\textit{S},3\textit{S})-3-Methylcyclopentanol}
\end{figure}

\begin{figure}[ht]
  \centering
  \smilesobabel[scale=0.7]{Cl/C=C/Br}{}
  \smilesobabel[scale=0.7]{Cl/C=C\Br}{}
  \caption{(\textit{E})-1-bromo-2-chloroethene \& (\textit{Z})-1-bromo-2-chloroethene}
\end{figure}

\begin{figure}[ht]
  \centering
  \chemobabel[scale=0.6]{draw/ATP.cdx}{-xd}
  \caption{ATP (Adenosine triphosphate)}
\end{figure}

\begin{figure}[ht]
  \centering
  \chemobabel[scale=0.4]{draw/Brevetoxin A.mol}{}
  \caption{Brevetoxin A}
\end{figure}

\end{document}