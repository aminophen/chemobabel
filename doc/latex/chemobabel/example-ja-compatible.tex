%#! 使い方:ファイル名が「example-ja.tex」のとき
%    1. platex example-ja.tex
%    2. pdflatex -shell-escape ChemFigFile.tex
%    3. platex example-ja.tex
%    4. dvipdfmx example-ja.dvi
% \smiles や \obabel のためのコマンドが含まれているので、必要に応じて削除してください:詳細はマクロ中の説明へ

\documentclass{jsarticle}
\usepackage[dvipdfmx]{graphicx}
\usepackage[dvipdfmx]{hyperref}
\usepackage{chemobabel}

%% ----- Optional: for compatibility -----
%% You can delete this region when you use \smilesobabel only (not \smiles and \obabel commands)
% ----- for \smiles command -----
\newcounter{smilescounter}
\setcounter{smilescounter}{1}
\newcommand{\smiles}[1]{
  \immediate\write18{obabel -:"#1" -O smilesimg\arabic{smilescounter}.png}
  \includegraphics{smilesimg\arabic{smilescounter}.png}
  \addtocounter{smilescounter}{1}
}
% ----- for \obabel command -----
\usepackage{graphvizObabel}
%% ----------

%! ----- For the purpose of extracting codes of all SMILES figures -----
%% Loading the compatible version...
%! ----- For the purpose of extracting all codes of Open Babel figures -----
%%
%% This is file `chemobabel-extract-comp.tex', bundled with `chemobabel.sty'.
%% 
%% Copyright 2014 Acetaminophen (Hironobu Yamashita)
%%   GitHub: https://github.com/aminophen
%%   Blog: http://acetaminophen.hatenablog.com/
%%
% Thanks: http://oku.edu.mie-u.ac.jp/tex/mod/forum/discuss.php?d=1411

%% Define intermediate output file and load packages
\newwrite\ChemFigFile
 \immediate\openout\ChemFigFile=ChemFigFile.tex\relax
 \immediate\write\ChemFigFile{\string\documentclass{article}}
 \immediate\write\ChemFigFile{\string\usepackage{graphicx}}
% for \smilesobabel and \chemobabel
 \immediate\write\ChemFigFile{\string\usepackage{chemobabel}}
%%

%% ----- Definition (Optional: for compatibility) -----
%% You can delete this region when you use \smilesobabel only (not \smiles and \obabel commands)
% for \smiles (Delete following 9 lines if you don't intend to use \smiles ; if you wish)
 \immediate\write\ChemFigFile{\string\newcounter{smilescounter}}
 \immediate\write\ChemFigFile{\string\setcounter{smilescounter}{1}}
 \immediate\write\ChemFigFile{%
  \string\newcommand{\string\smiles}[1]{%
   \string\immediate\string\write18{obabel -:"\string#1" -O smilesimg\string\arabic{smilescounter}.png}
   \string\includegraphics{smilesimg\string\arabic{smilescounter}.png}
   \string\addtocounter{smilescounter}{1}%
  }%
 }
% for \obabel (Delete the next line if you don't intend to use `graphvizObabel.sty'; necessary if you don't have `graphvizObabel.sty')
 \immediate\write\ChemFigFile{\string\usepackage{graphvizObabel}}
%% ----------

%% Read and write
 \immediate\write\ChemFigFile{\string\begin{document}}
  \immediate\write\ChemFigFile{\string\begingroup}
  \immediate\write\ChemFigFile{\string\catcode`\string\~=0}
  \immediate\write\ChemFigFile{\string\catcode`\string\%=11}
  \immediate\write\ChemFigFile{\string\catcode`\string\\=11}
 \AtEndDocument{%
  \immediate\write\ChemFigFile{\string~endgroup}
  \immediate\write\ChemFigFile{\string\end{document}}%
  \immediate\closeout\ChemFigFile%
 }
% Support for SMILES notations including "backslash" and "percent"
% for \smilesobabel (~ is used as an escape character instead of \)
 \let\oldsmilesobabel=\smilesobabel
 \renewcommand\smilesobabel[3][scale=1]{%
  \immediate\write\ChemFigFile{%
   \string~smilesobabel[#1]{#2}{#3}%
   \string~newpage}%
  \oldsmilesobabel[#1]{#2}{#3}}
% for \chemobabel (~ is used as an escape character instead of \)
 \let\oldchemobabel=\chemobabel
 \renewcommand\chemobabel[3][scale=1]{%
  \immediate\write\ChemFigFile{%
   \string~chemobabel[#1]{#2}{#3}%
   \string~newpage}%
  \oldchemobabel[#1]{#2}{#3}}
%%

%% ----- Read and write (Optional: for compatibility) -----
%% You can delete this region when you use \smilesobabel only (not \smiles and \obabel commands)
% for \smiles
 \let\oldsmiles=\smiles
 \def\smiles#1{%
  \immediate\write\ChemFigFile{%
   \string~smiles{#1}%
   \string~newpage}%
  \oldsmiles{#1}}
% for \obabel
 \let\oldobabel=\obabel
 \def\obabel[#1]#2{%
  \immediate\write\ChemFigFile{%
   \string~obabel[#1]{#2}%
   \string~newpage}%
  \oldobabel[#1]{#2}}
%% ----------

% ----- [EOF] -----

\title{\textsf{chemobabel.sty} の使用例(互換バージョン)}
\author{アセトアミノフェン}
\begin{document}
\maketitle
以下の構造式は、\textsf{chemobabel} パッケージを用いて Open Babel と Inkscape を介した自動変換により生成しました。
最後の2つは古い \verb|\smiles| と \verb|\obabel| による出力です。

\begin{figure}[ht]
  \centering
  \smilesobabel{CCO}{}
  \caption{エタノール}
\end{figure}

\begin{figure}[ht]
  \centering
  \smilesobabel[scale=0.8]{C1[C@H](C)C[C@@H](O)C1}{}
  \caption{(1\textit{S},3\textit{S})-3-メチルシクロペンタノール}
\end{figure}

\begin{figure}[ht]
   \centering
  \begingroup
    \catcode`\~=0
    \catcode`\%=11
    \catcode`\\=11
    ~smilesobabel[scale=0.7]{Cl/C=C/Br}{}
    ~smilesobabel[scale=0.7]{Cl/C=C\Br}{}
  ~endgroup
  \caption{(\textit{E})-1-ブロモ-2-クロロエテンと(\textit{Z})-1-ブロモ-2-クロロエテン}
\end{figure}

\begin{figure}[ht]
  \centering
  \chemobabel[scale=0.6]{draw/ATP.cdx}{-xd}
  \caption{ATP (アデノシン三リン酸)}
\end{figure}

\begin{figure}[ht]
  \centering
  \chemobabel[scale=0.4]{draw/Brevetoxin A.mol}{}
  \caption{ブレベトキシンA}
\end{figure}

\begin{figure}[ht]
  \centering
  \smiles{C1[C@H](C)C[C@@H](O)C1}
  \caption{(1\textit{S},3\textit{S})-3-メチルシクロペンタノール}
\end{figure}

\begin{figure}[ht]
  \centering
  \obabel[scale=0.8]{C1[C@H](C)C[C@@H](O)C1}
  \caption{(1\textit{S},3\textit{S})-3-メチルシクロペンタノール}
\end{figure}

\end{document}