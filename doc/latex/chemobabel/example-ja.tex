%#! 使い方:ファイル名が「example-ja.tex」のとき
%    1. platex example-ja.tex
%    2. pdflatex -shell-escape ChemFigFile.tex
%    3. platex example-ja.tex
%    4. dvipdfmx example-ja.dvi

\documentclass{jsarticle}
\usepackage[dvipdfmx]{graphicx}
\usepackage[dvipdfmx]{hyperref}
\usepackage{chemobabel}

%! ----- For the purpose of extracting codes of all SMILES figures -----
%! ----- For the purpose of extracting all codes of Open Babel figures -----
%%
%% This is file `chemobabel-extract.tex', bundled with `chemobabel.sty'.
%% 
%% Copyright 2014-2016 Acetaminophen (Hironobu Yamashita)
%%   GitHub: https://github.com/aminophen
%%   Blog: http://acetaminophen.hatenablog.com/
%%
%% Thanks: http://oku.edu.mie-u.ac.jp/tex/mod/forum/discuss.php?d=1411
%

\NeedsTeXFormat{LaTeX2e}
\ProvidesFile{chemobabel-extract.tex}
  [2016/03/07 v0.9e Support file for chemobabel.sty]

%% Define intermediate output file and load packages
\newwrite\ChemFigFile
\immediate\openout\ChemFigFile=ChemFigFile.tex\relax
\immediate\write\ChemFigFile{\string\documentclass{article}}
\immediate\write\ChemFigFile{\string\usepackage{graphicx}}
% for \smilesobabel and \chemobabel
\immediate\write\ChemFigFile{\string\usepackage{chemobabel}}
%%

%% Read and write
\immediate\write\ChemFigFile{\string\begin{document}}
\AtEndDocument{%
  \immediate\write\ChemFigFile{\string\end{document}}%
  \immediate\closeout\ChemFigFile%
}
\renewcommand\smilesobabel{%
  \begingroup
  \let\do\@makeother
  \dospecials
  \catcode`\{=1
  \catcode`\}=2
  \@@smilesobabel
}
\newcommand\@@smilesobabel[3][scale=1]{%
  \endgroup
  \immediate\write\ChemFigFile{%
    \string\smilesobabel[#1]{#2}{#3}%
    \string\newpage}%
  [\smilesob@belGetName.\chemob@belimgExt]%
  \addtocounter{smilesob@belCounter}{1}}
\renewcommand\chemobabel{%
  \begingroup
  \let\do\@makeother
  \dospecials
  \catcode`\{=1
  \catcode`\}=2
  \@@chemobabel
}
\newcommand\@@chemobabel[3][scale=1]{%
  \endgroup
  \immediate\write\ChemFigFile{%
    \string\chemobabel[#1]{#2}{#3}%
    \string\newpage}%
  [\chemob@belGetName.\chemob@belimgExt]%
  \addtocounter{chemob@belCounter}{1}}
%%

% ----- [EOF] -----


\title{\textsf{chemobabel.sty} の使用例}
\author{アセトアミノフェン}
\begin{document}
\maketitle
以下の構造式は、\textsf{chemobabel} パッケージを用いて Open Babel と Inkscape を介した自動変換により生成しました。

\begin{figure}[ht]
  \centering
  \smilesobabel{CCO}{}
  \caption{エタノール}
\end{figure}

\begin{figure}[ht]
  \centering
  \smilesobabel[scale=0.8]{C1[C@H](C)C[C@@H](O)C1}{}
  \caption{(1\textit{S},3\textit{S})-3-メチルシクロペンタノール}
\end{figure}

\begin{figure}[ht]
   \centering
  \begingroup
    \catcode`\~=0
    \catcode`\%=11
    \catcode`\\=11
    ~smilesobabel[scale=0.7]{Cl/C=C/Br}{}
    ~smilesobabel[scale=0.7]{Cl/C=C\Br}{}
  ~endgroup
  \caption{(\textit{E})-1-ブロモ-2-クロロエテンと(\textit{Z})-1-ブロモ-2-クロロエテン}
\end{figure}

\begin{figure}[ht]
  \centering
  \chemobabel[scale=0.6]{draw/ATP.cdx}{-xd}
  \caption{ATP (アデノシン三リン酸)}
\end{figure}

\begin{figure}[ht]
  \centering
  \chemobabel[scale=0.4]{draw/Brevetoxin A.mol}{}
  \caption{ブレベトキシンA}
\end{figure}

\end{document}