%#! Usage: (suppose this source file = `example.tex')
%    1. pdflatex example.tex
%    2. pdflatex -shell-escape ChemFigFile.tex
%    3. pdflatex example.tex

\documentclass{article}
\usepackage{graphicx}
\usepackage{hyperref}
\usepackage[extract]{chemobabel} % For the purpose of extracting codes of all chemical structural formulas (new feature in ver.0.5)

\title{An Example for \textsf{chemobabel} Package}
\author{Acetaminophen}
\begin{document}
\maketitle

The formulas below are generated using \textsf{chemobabel} package with the help of Open Babel and Inkscape.

\begin{figure}[ht]
  \centering
  \smilesobabel{CCO}{}
  \caption{Ethanol}
\end{figure}

\begin{figure}[ht]
  \centering
  \smilesobabel[scale=0.8]{C1[C@H](C)C[C@@H](O)C1}{}
  \caption{(1\textit{S},3\textit{S})-3-Methylcyclopentanol}
\end{figure}

\begin{figure}[ht]
   \centering
  \begingroup
    \catcode`\~=0
    \catcode`\%=11
    \catcode`\\=11
    ~smilesobabel[scale=0.7]{Cl/C=C/Br}{}
    ~smilesobabel[scale=0.7]{Cl/C=C\Br}{}
  ~endgroup
  \caption{(\textit{E})-1-bromo-2-chloroethene \& (\textit{Z})-1-bromo-2-chloroethene}
\end{figure}

\begin{figure}[ht]
  \centering
  \chemobabel[scale=0.6]{draw/ATP.cdx}{-xd}
  \caption{ATP (Adenosine triphosphate)}
\end{figure}

\begin{figure}[ht]
  \centering
  \chemobabel[scale=0.4]{draw/Brevetoxin A.mol}{}
  \caption{Brevetoxin A}
\end{figure}

\end{document}