%#! 使い方:ファイル名が「example-ja.tex」のとき
%    1. platex example-ja.tex
%    2. pdflatex -shell-escape ChemFigFile.tex
%    3. platex example-ja.tex
%    4. dvipdfmx example-ja.dvi

\documentclass{jsarticle}
\usepackage[dvipdfmx]{graphicx}
\usepackage[dvipdfmx]{hyperref}
\usepackage[extract]{chemobabel} % すべての化学構造式に関するコマンドを抽出(ver.0.5の新機能)

\title{\textsf{chemobabel.sty} の使用例}
\author{アセトアミノフェン}
\begin{document}
\maketitle
以下の構造式は、\textsf{chemobabel} パッケージを用いて Open Babel と Inkscape を介した自動変換により生成しました。

\begin{figure}[ht]
  \centering
  \smilesobabel{CCO}{}
  \caption{エタノール}
\end{figure}

\begin{figure}[ht]
  \centering
  \smilesobabel[scale=0.8]{C1[C@H](C)C[C@@H](O)C1}{}
  \caption{(1\textit{S},3\textit{S})-3-メチルシクロペンタノール}
\end{figure}

\begin{figure}[ht]
   \centering
  \begingroup
    \catcode`\~=0
    \catcode`\%=11
    \catcode`\\=11
    ~smilesobabel[scale=0.7]{Cl/C=C/Br}{}
    ~smilesobabel[scale=0.7]{Cl/C=C\Br}{}
  ~endgroup
  \caption{(\textit{E})-1-ブロモ-2-クロロエテンと(\textit{Z})-1-ブロモ-2-クロロエテン}
\end{figure}

\begin{figure}[ht]
  \centering
  \chemobabel[scale=0.6]{draw/ATP.cdx}{-xd}
  \caption{ATP (アデノシン三リン酸)}
\end{figure}

\begin{figure}[ht]
  \centering
  \chemobabel[scale=0.4]{draw/Brevetoxin A.mol}{}
  \caption{ブレベトキシンA}
\end{figure}

\end{document}